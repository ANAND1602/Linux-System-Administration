%% packages
%\usepackage{microtype}
\usepackage{enumerate} % needed for numeric ordered indexing
\usepackage{array}	% needed for \bfseries in tables
\usepackage{booktabs} % needed for top/mid/bottomrule in tables
\usepackage{longtable} % this and next 2 pkgs allow tables to flow over multiple pages
\usepackage{tabularx}
\usepackage{ltablex}
\usepackage{tocloft} % address issue where space between table of contents index and section/subsection title becomes 0 spaces
\setlength{\cftsecnumwidth}{3em} % set above space to 3em for section headings
\setlength{\cftsubsecnumwidth}{4em} % set above space to 4em for subsection headings
\usepackage{indentfirst}
%\usepackage{minted}
\usepackage{textcomp}    % also needed for upquotes in lstlisting
\usepackage{color}
\usepackage[parfill]{parskip} % Use for no indent.
\usepackage{float} % needed for [H] place here option for figures
\usepackage{scrextend} % needed for indenting block of text
\usepackage{blindtext} % needed for creating dummy text passages
\usepackage{xparse,nameref}
\usepackage[french,main=english]{babel} % if you didn't specify main=english, chapter appears as chapitre in chapter headings and autorefs
\usepackage{amsmath} % very little math in this document
\usepackage{soul} % for hl command (highlighting with yellow background)
\usepackage{upquote}
%\usepackage[utf8]{inputenc}
\usepackage[T1]{fontenc} % needed for upquotes in lstlisting and accented characters, XeLaTeX or LuaLaTeX can accpet UTF-8 input natively.
\usepackage{multirow}
\usepackage{fancyvrb}
\usepackage{alltt}
%% todo

% With many thanks to MLC for this todo section. Modified slightly: added [inline] and had to define colors.
% http://tex.stackexchange.com/questions/9796/how-to-add-todo-notes
\usepackage{xargs}
\usepackage[pdftex,dvipsnames]{xcolor} % Coloured text.
\colorlet{OLIVEGREEN}{OliveGreen}
\colorlet{PROCESSBLUE}{ProcessBlue}
\usepackage[colorinlistoftodos,prependcaption,textsize=tiny]{todonotes}
\newcommandx{\howto}[2][1=]{\todo[inline,linecolor=red,backgroundcolor=red!25,bordercolor=red,#1]{#2}}
\newcommandx{\change}[2][1=]{\todo[inline,linecolor=blue,backgroundcolor=blue!25,bordercolor=blue,#1]{#2}}
\newcommandx{\add}[2][1=]{\todo[inline,linecolor=ProcessBlue,backgroundcolor=ProcessBlue!25,bordercolor=ProcessBlue,#1]{#2}}
\newcommandx{\find}[2][1=]{\todo[inline,linecolor=OliveGreen,backgroundcolor=OliveGreen!25,bordercolor=OliveGreen,#1]{#2}}
\newcommandx{\improve}[2][1=]{\todo[inline,linecolor=Plum,backgroundcolor=Plum!25,bordercolor=Plum,#1]{#2}}
\newcommandx{\why}[2][1=]{\todo[inline,linecolor=violet,backgroundcolor=violet!25,bordercolor=violet,#1]{#2}}
\newcommandx{\donotshow}[2][1=]{\todo[disable,#1]{#2}}

%% listing package --- the most package for displaying code, supports highlighting of all the most common languages and it is highly customizable
\usepackage{listings}
\lstset{
	keywords={tar,gzip},
	%backgroundcolor=\color{lbcolor},
	%tabsize=4,
	rulecolor=,
	language=bash,
	%basicstyle=\scriptsize,
	basicstyle=\ttfamily,
	upquote=true,
	aboveskip={1.5\baselineskip},
	%columns=fixed,
	showstringspaces=false,
	extendedchars=true,
	breaklines=true,
	%prebreak = \raisebox{0ex}[0ex][0ex]{\ensuremath{\hookleftarrow}},
	prebreak={},
	%breakatwhitespace=false,
	frame=single,
	%numbers=left,
	%stepnumber=1,
	%numbersep=10pt,
	%numberstyle=footnotesize,
	showtabs=false,
	showspaces=false,
	showstringspaces=false,
	identifierstyle=\ttfamily,
	keywordstyle=\color[rgb]{0,0,1},
	%keywordstyle=\bfseries\ttfamily\color[rgb]{0,0,1},
	%commentstyle=\color[rgb]{0.133,0.545,0.133},
	commentstyle=\color[rgb]{0.133,0.545,0.133}\fontseries{lc}\selectfont\itshape,columns=fullflexible,
	stringstyle=\color[rgb]{0.627,0.126,0.941},
	%columns=flexible,texcl
}

%% miscellaneous paging settings
\usepackage{changepage} % needed for adjustwidth
\usepackage{lastpage} % if you want footer page of pages
\usepackage{xspace} % for \xspace, one space

%% hyperlinks
\usepackage[english, % have to specify english or autorefs will be in french because of using the babel package
	%pdftex, % 
	colorlinks=true % false would place a box around the word instead of coloring the word
	,breaklinks % allows a link description to flow over several lines
	%,ngerman
	]{hyperref} % needed for creating hyperlinks in the document, the option colorlinks=true gets rid of the awful boxes, breaklinks breaks lonkg links (list of figures), and ngerman sets everything for german as default hyperlinks language
	
%\usepackage[hyphenbreaks]{breakurl} % needed for breaking of URLs in literature references, hyphenbreaks even at the left going through a page (with hypenation)
%%

%% Mauricio did not like the default hyperlink, citation, or url colors so he designed his own. In order to do this you need the 'xcolor' package.

%% custom colours
\definecolor{c1}{rgb}{0,0,1} % blue
\definecolor{c2}{rgb}{0,0.3,0.9} % light blue
\definecolor{c3}{rgb}{0.3,0,0.9} % red blue

%% hyperlink colours
\hypersetup{
    linkcolor={c1}, % internal links
    citecolor={c2}, % citations
    urlcolor={c3} % external links/urls
}

%% Mauricio likes the citation format 'Author (year)'. You need the 'natbib' package for this. That is why the 'cite' package is uncommented.
%\usepackage{cite} % needed for cite

%% Miscellaneous packages
\usepackage[round,authoryear]{natbib} % needed for cite and abbrvnat bibliography style
\usepackage[nottoc]{tocbibind} % needed for displaying bibliography and other in the table of contents
\usepackage{graphicx} % needed for \includegraphics
\usepackage{media9}
\usepackage{multimedia} % for embedding videos

\usepackage{longtable} % needed for long tables over pages
\usepackage{bigstrut} % needed for the command \bigstrut
\usepackage{enumitem} % needed for some options in enumerate and indent
\usepackage{todonotes} % needed for todos

% Create Index
%\usepackage{nomencl}
%\makenomenclature
% user0:Make Nomenclature makeindex -s nomencl.ist -t %.nlg -o %.nls %.nlo
\usepackage{makeidx}
\makeindex
%\usepackage[splitindex]{imakeidx}
%\makeindex[options=-s latex.ist -c, intoc, columns=3]
%\makeindex
%\usepackage[splitindex]{imakeidx} % needed for creating an index
%\makeindex[name=style, title=Index of styles, options=-s latex.ist -c, intoc, columns=3]